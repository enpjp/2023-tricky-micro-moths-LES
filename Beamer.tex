% Options for packages loaded elsewhere
\PassOptionsToPackage{unicode}{hyperref}
\PassOptionsToPackage{hyphens}{url}
%
\documentclass[
  ignorenonframetext,
]{beamer}
\usepackage{pgfpages}
\setbeamertemplate{caption}[numbered]
\setbeamertemplate{caption label separator}{: }
\setbeamercolor{caption name}{fg=normal text.fg}
\beamertemplatenavigationsymbolsempty
% Prevent slide breaks in the middle of a paragraph
\widowpenalties 1 10000
\raggedbottom
\setbeamertemplate{part page}{
  \centering
  \begin{beamercolorbox}[sep=16pt,center]{part title}
    \usebeamerfont{part title}\insertpart\par
  \end{beamercolorbox}
}
\setbeamertemplate{section page}{
  \centering
  \begin{beamercolorbox}[sep=12pt,center]{part title}
    \usebeamerfont{section title}\insertsection\par
  \end{beamercolorbox}
}
\setbeamertemplate{subsection page}{
  \centering
  \begin{beamercolorbox}[sep=8pt,center]{part title}
    \usebeamerfont{subsection title}\insertsubsection\par
  \end{beamercolorbox}
}
\AtBeginPart{
  \frame{\partpage}
}
\AtBeginSection{
  \ifbibliography
  \else
    \frame{\sectionpage}
  \fi
}
\AtBeginSubsection{
  \frame{\subsectionpage}
}
\usepackage{amsmath,amssymb}
\usepackage{iftex}
\ifPDFTeX
  \usepackage[T1]{fontenc}
  \usepackage[utf8]{inputenc}
  \usepackage{textcomp} % provide euro and other symbols
\else % if luatex or xetex
  \usepackage{unicode-math} % this also loads fontspec
  \defaultfontfeatures{Scale=MatchLowercase}
  \defaultfontfeatures[\rmfamily]{Ligatures=TeX,Scale=1}
\fi
\usepackage{lmodern}
\usetheme[]{AnnArbor}
\usecolortheme{dolphin}
\usefonttheme{structurebold}
\ifPDFTeX\else
  % xetex/luatex font selection
\fi
% Use upquote if available, for straight quotes in verbatim environments
\IfFileExists{upquote.sty}{\usepackage{upquote}}{}
\IfFileExists{microtype.sty}{% use microtype if available
  \usepackage[]{microtype}
  \UseMicrotypeSet[protrusion]{basicmath} % disable protrusion for tt fonts
}{}
\makeatletter
\@ifundefined{KOMAClassName}{% if non-KOMA class
  \IfFileExists{parskip.sty}{%
    \usepackage{parskip}
  }{% else
    \setlength{\parindent}{0pt}
    \setlength{\parskip}{6pt plus 2pt minus 1pt}}
}{% if KOMA class
  \KOMAoptions{parskip=half}}
\makeatother
\usepackage{xcolor}
\newif\ifbibliography
\setlength{\emergencystretch}{3em} % prevent overfull lines
\providecommand{\tightlist}{%
  \setlength{\itemsep}{0pt}\setlength{\parskip}{0pt}}
\setcounter{secnumdepth}{-\maxdimen} % remove section numbering

\usepackage{amsmath}
\usepackage{nomencl}
\makenomenclature 


%\usepackage{booktabs}
%\usepackage{longtable}
%\usepackage{morefloats}
%\extrafloats{100}
%\date{August 25, 2021}
%\renewcommand{\today}{September 5, 2021}

% Set the copyright footer
%\lfoot{\copyright 2021 P.J. Palmer  P.M. Leonard}

% Some figure placement options.
%\usepackage[figuresonly,nomarkers,fighead, figlist]{endfloat}
%\usepackage[figuresonly,nomarkers,nolists]{endfloat}
% Put multiple figures per page
%\renewcommand{\efloatseparator}{\mbox{}}
\ifLuaTeX
  \usepackage{selnolig}  % disable illegal ligatures
\fi
\IfFileExists{bookmark.sty}{\usepackage{bookmark}}{\usepackage{hyperref}}
\IfFileExists{xurl.sty}{\usepackage{xurl}}{} % add URL line breaks if available
\urlstyle{same}
\hypersetup{
  pdftitle={A nice presentation},
  pdfauthor={Paul J. Palmer},
  hidelinks,
  pdfcreator={LaTeX via pandoc}}

\title{A nice presentation}
\author{Paul J. Palmer}
\date{}

\begin{document}
\frame{\titlepage}

\begin{frame}{Introduction}
\protect\hypertarget{introduction}{}
Identifying micromoths can be a challenging task due to their small size
and often subtle differences in appearance. Micromoths belong to the
microlepidoptera group, which includes many small moths. Here are some
general tips to help you identify these tricky little insects:
\end{frame}

\begin{frame}{Size}
\protect\hypertarget{size}{}
\begin{itemize}
\tightlist
\item
  Micromoths are generally small, with wingspans usually less than 20 mm
  (0.79 inches).
\item
  Keep in mind that size alone is not a definitive characteristic, as
  there are larger moths that might still be considered
  microlepidoptera.
\end{itemize}
\end{frame}

\begin{frame}{Wing Shape and Venation:}
\protect\hypertarget{wing-shape-and-venation}{}
\begin{itemize}
\tightlist
\item
  Pay attention to the shape of the wings and their venation (pattern of
  veins). Different micromoth species may have distinct wing shapes and
  vein patterns.
\item
  Some micromoths have narrow, pointed wings, while others may have
  broader wings.
\end{itemize}
\end{frame}

\begin{frame}{Coloration and Patterns:}
\protect\hypertarget{coloration-and-patterns}{}
\begin{itemize}
\tightlist
\item
  Observe the coloration and patterns on the wings. Micromoths can have
  intricate patterns, even though they might be subtle.
\item
  Look for distinctive markings such as spots, lines, or bands on the
  wings.
\end{itemize}
\end{frame}

\begin{frame}{Antennae:}
\protect\hypertarget{antennae}{}
\begin{itemize}
\tightlist
\item
  Examine the antennae. The shape and characteristics of the antennae
  can be useful for identification.
\item
  Micromoths may have thread-like or feathery antennae.
\end{itemize}
\end{frame}

\begin{frame}{Resting Posture:}
\protect\hypertarget{resting-posture}{}
\begin{itemize}
\tightlist
\item
  Take note of the moth's resting posture. Some micromoths hold their
  wings flat, while others may fold their wings around their bodies.
\end{itemize}
\end{frame}

\begin{frame}{Habitat and Behavior:}
\protect\hypertarget{habitat-and-behavior}{}
\begin{itemize}
\tightlist
\item
  Consider the habitat in which you find the moth. Different micromoth
  species may prefer specific environments.
\item
  Note the behavior of the moth, such as its flight pattern and feeding
  habits.
\end{itemize}
\end{frame}

\begin{frame}{\textbf{Use a Field Guide:}}
\protect\hypertarget{use-a-field-guide}{}
\begin{itemize}
\tightlist
\item
  A regional field guide to moths and butterflies can be a valuable
  resource. These guides often provide images, descriptions, and
  information on distribution.
\item
  Online resources and mobile apps dedicated to moth identification can
  also be helpful.
\end{itemize}
\end{frame}

\begin{frame}{Photography:}
\protect\hypertarget{photography}{}
\begin{itemize}
\tightlist
\item
  If possible, take clear photographs of the micromoth from various
  angles. This can be useful for later reference or for seeking
  assistance from experts or online communities dedicated to moth
  identification.
\end{itemize}
\end{frame}

\begin{frame}{Checklist of Common Micromoth Families:}
\protect\hypertarget{checklist-of-common-micromoth-families}{}
\begin{itemize}
\tightlist
\item
  Familiarize yourself with common micromoth families in your region.
  This can help narrow down possibilities.
\end{itemize}

Remember that micromoth identification can be challenging, and sometimes
it may require microscopic examination of specific features. If you're
having difficulty, reaching out to entomologists, local insect groups,
or online forums dedicated to moth identification can provide valuable
assistance.
\end{frame}

\end{document}
